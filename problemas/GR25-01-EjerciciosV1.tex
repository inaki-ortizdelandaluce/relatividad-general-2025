% --------------------------------------------------------------
% This is all preamble stuff that you don't have to worry about.
% Head down to where it says "Start here"
% --------------------------------------------------------------
 
\documentclass[12pt]{article}
 
\usepackage[margin=1in]{geometry} 
\usepackage{amsmath,amsthm,amssymb}
\usepackage[spanish]{babel} 
\usepackage[T1]{fontenc}

\newcommand{\N}{\mathbb{N}}
\newcommand{\Z}{\mathbb{Z}}
 
\newenvironment{theorem}[2][Teorema]{\begin{trivlist} 
\item[\hskip \labelsep {\bfseries #1}\hskip \labelsep {\bfseries #2.}]}{\end{trivlist}}
\newenvironment{lemma}[2][Lema]{\begin{trivlist} 
\item[\hskip \labelsep {\bfseries #1}\hskip \labelsep {\bfseries #2.}]}{\end{trivlist}}
\newenvironment{exercise}[2][Ejercicio]{\begin{trivlist} 
\item[\hskip \labelsep {\bfseries #1}\hskip \labelsep {\bfseries #2.}]}{\end{trivlist}}
\newenvironment{problem}[2][Problema]{\begin{trivlist} 
\item[\hskip \labelsep {\bfseries #1}\hskip \labelsep {\bfseries #2.}]}{\end{trivlist}}
\newenvironment{question}[2][Pregunta]{\begin{trivlist} 
\item[\hskip \labelsep {\bfseries #1}\hskip \labelsep {\bfseries #2.}]}{\end{trivlist}}
\newenvironment{corollary}[2][Corolario]{\begin{trivlist} 
\item[\hskip \labelsep {\bfseries #1}\hskip \labelsep {\bfseries #2.}]}{\end{trivlist}}

\newenvironment{solution}{\begin{proof}[Solución]}{\end{proof}} 
 
\begin{document}
 
% --------------------------------------------------------------
%                         Start here
% --------------------------------------------------------------
 
\title{Ejercicios de Relatividad General y Cosmología}
\author{Iñaki Ortiz de Landaluce\\ 
Introducción a la Relatividad General y Cosmología, Curso 2025-2026}
\date{} 
\maketitle

% ==============================================================
%                     UNIDAD 1: INTRODUCCIÓN
% ==============================================================
\section*{Unidad 1 Introducción a la Relatividad}

\begin{exercise}{1.1} Siendo $\gamma$ el factor de Lorentz, la cantidad $(\gamma -1)$ da una medida de la diferencia entre los efectos relativistas y la mecánica Newtoniana para distintos regímenes de velocidades. Siendo $\beta=v/c$, calcula su valor para obtener los siguentes valores de $(\gamma-1)$: (a) 0.01, (b) 0.1, (c) 1 (d) 10 y (e) 100.
\end{exercise}

\begin{exercise}{1.2} Una varilla de longitud $1 \text{m}$ está inclinada $45^\circ$ en el plano $xy$ con respecto al eje $x$. Un observador con velocidad $\sqrt{2/3}c$ se aproxima a la varilla en la dirección positiva del eje $x$. ¿Cuál es la longitud de la varilla y el ángulo de inclinación con respecto a su eje $x$ que mide el observador?
\end{exercise}

\begin{exercise}{1.3} Cuando los rayos cósmicos primarios impactan en la atmósfera, se crean muones a una altitud entre $10 \text{km}$ y $20 \text{km}$. Un muón en el laboratorio vive en promedio un tiempo $\tau_0 = 2.2 \cdot 10^{-6} \text{s}$ antes de desintegrarse en un electrón (o un positrón) y dos neutrinos. Aunque un muón sólo puede moverse $\tau_0 c \approx 660 \text{m}$ durante el tiempo $\tau_0$, una gran fracción de muones logra alcanzar la superficie de la Tierra. ¿Cómo puede explicarse esto? Realice un cálculo numérico para un muón que se mueve con velocidad $0.999c$.
\end{exercise}

\begin{exercise}{1.4} 
Tenemos dos sistemas de referencia inerciales $S$ y $S'$, donde $S'$ se mueve con velocidad $v$ en la dirección del eje $x$ positivo respecto a $S$. Si un objeto se mueve con velocidad constante $u$ respecto a $S$ a lo largo del mismo eje x, demuestra que la velocidad medida desde el sistema de referencia $S'$ a lo largo del eje $x'$, satisface la siguiente ecuación:
\[
u' = \frac{u - v}{1 - \frac{u v}{c^2}}
\]
\end{exercise}

\begin{exercise}{1.5} 
Una varilla se mueve con velocidad $v$ a lo largo del eje $x$ positivo en un sistema inercial $S$
$S$. Un observador en reposo en $S'$ mide que la longitud de la varilla es $L$. Otro observador se 
mueve con velocidad $-v$ a lo largo del eje $x$. ¿Qué longitud, expresada como función de $L$ y $v$, medirá este observador para la varilla? La medición se realiza de la manera habitual, midiendo los extremos de forma simultánea para cada observador en sus respectivos sistemas de referencia.
\end{exercise}

 
\end{document}