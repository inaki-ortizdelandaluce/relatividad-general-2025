% --------------------------------------------------------------
% This is all preamble stuff that you don't have to worry about.
% Head down to where it says "Start here"
% --------------------------------------------------------------
 
\documentclass[12pt]{article}
 
\usepackage[margin=1in]{geometry} 
\usepackage{amsmath,amsthm,amssymb}
\usepackage[spanish]{babel} % Added for Spanish hyphenation and names
\usepackage[T1]{fontenc}

\newcommand{\N}{\mathbb{N}}
\newcommand{\Z}{\mathbb{Z}}
 
\newenvironment{theorem}[2][Teorema]{\begin{trivlist} % Changed to Spanish
\item[\hskip \labelsep {\bfseries #1}\hskip \labelsep {\bfseries #2.}]}{\end{trivlist}}
\newenvironment{lemma}[2][Lema]{\begin{trivlist} % Changed to Spanish
\item[\hskip \labelsep {\bfseries #1}\hskip \labelsep {\bfseries #2.}]}{\end{trivlist}}
\newenvironment{exercise}[2][Ejercicio]{\begin{trivlist} % Changed to Spanish
\item[\hskip \labelsep {\bfseries #1}\hskip \labelsep {\bfseries #2.}]}{\end{trivlist}}
\newenvironment{problem}[2][Problema]{\begin{trivlist} % Changed to Spanish
\item[\hskip \labelsep {\bfseries #1}\hskip \labelsep {\bfseries #2.}]}{\end{trivlist}}
\newenvironment{question}[2][Pregunta]{\begin{trivlist} % Changed to Spanish
\item[\hskip \labelsep {\bfseries #1}\hskip \labelsep {\bfseries #2.}]}{\end{trivlist}}
\newenvironment{corollary}[2][Corolario]{\begin{trivlist} % Changed to Spanish
\item[\hskip \labelsep {\bfseries #1}\hskip \labelsep {\bfseries #2.}]}{\end{trivlist}}

\newenvironment{solution}{\begin{proof}[Solución]}{\end{proof}} % Changed to Spanish
 
\begin{document}
 
% --------------------------------------------------------------
%                         Start here
% --------------------------------------------------------------
 
\title{Ejercicios de Relatividad General y Cosmología}
\author{Iñaki Ortiz de Landaluce\\ %replace with your name
Introducción a la Relatividad General y Cosmología}

\maketitle

% ==============================================================
%                     UNIDAD 1: INTRODUCCIÓN
% ==============================================================
\section*{Unidad 1 Introducción a la Relatividad}
 
\begin{exercise}{1.1} Una varilla de longitud $1 \text{m}$ está inclinada $45^\circ$ en el plano $xy$ con respecto al eje $x$. Un observador con velocidad $\sqrt{2/3}c$ se aproxima a la varilla en la dirección positiva del eje $x$. ¿Cuál es la longitud de la varilla y el ángulo de inclinación con respecto a su eje $x$ que mide el observador?
\end{exercise}

\begin{exercise}{1.2} Cuando los rayos cósmicos primarios impactan en la atmósfera, se crean muones a una altitud entre $10 \text{km}$ y $20 \text{km}$. Un muón en el laboratorio vive en promedio un tiempo $\tau_0 = 2.2 \cdot 10^{-6} \text{s}$ antes de desintegrarse en un electrón (o un positrón) y dos neutrinos. Aunque un muón sólo puede moverse $\tau_0 c \approx 660 \text{m}$ durante el tiempo $\tau_0$, una gran fracción de muones logra alcanzar la superficie de la Tierra. ¿Cómo puede explicarse esto? Realice un cálculo numérico para un muón que se mueve con velocidad $0.999c$.
\end{exercise}

% You can add more exercises for Unit 1 here
%\begin{exercise}{1.2}
%New exercise text for Unit 1...
%\end{exercise}

 
\end{document}