% --------------------------------------------------------------
% This is all preamble stuff that you don't have to worry about.
% Head down to where it says "Start here"
% --------------------------------------------------------------
 
\documentclass[12pt]{article}
 
\usepackage[margin=1in]{geometry} 
\usepackage{amsmath,amsthm,amssymb}
\usepackage[spanish]{babel} 
\usepackage[T1]{fontenc}
\usepackage{tikz}
\usepackage{pgfplots}
\pgfplotsset{compat=1.18}
\usetikzlibrary{arrows.meta}
\usepackage[labelformat=empty]{caption}
\usepackage{float}

\newcommand{\N}{\mathbb{N}}
\newcommand{\Z}{\mathbb{Z}}
 
\newenvironment{theorem}[2][Teorema]{\begin{trivlist} 
\item[\hskip \labelsep {\bfseries #1}\hskip \labelsep {\bfseries #2.}]}{\end{trivlist}}
\newenvironment{lemma}[2][Lema]{\begin{trivlist} 
\item[\hskip \labelsep {\bfseries #1}\hskip \labelsep {\bfseries #2.}]}{\end{trivlist}}
\newenvironment{exercise}[2][Ejercicio]{\begin{trivlist} 
\item[\hskip \labelsep {\bfseries #1}\hskip \labelsep {\bfseries #2.}]}{\end{trivlist}}
\newenvironment{problem}[2][Problema]{\begin{trivlist} 
\item[\hskip \labelsep {\bfseries #1}\hskip \labelsep {\bfseries #2.}]}{\end{trivlist}}
\newenvironment{question}[2][Pregunta]{\begin{trivlist} 
\item[\hskip \labelsep {\bfseries #1}\hskip \labelsep {\bfseries #2.}]}{\end{trivlist}}
\newenvironment{corollary}[2][Corolario]{\begin{trivlist} 
\item[\hskip \labelsep {\bfseries #1}\hskip \labelsep {\bfseries #2.}]}{\end{trivlist}}

\newenvironment{solution}{\begin{proof}[Solución]}{\end{proof}} 
 
\begin{document}
 
% --------------------------------------------------------------
%                         Start here
% --------------------------------------------------------------
 
\title{\textbf{Ejercicios}}
\author{\textbf{Introducción a la Relatividad General y la Cosmología}\\
\textbf{Curso 2025-2026}}\date{} 
\maketitle

% ==============================================================
%                     UNIDAD 1: INTRODUCCIÓN
% ==============================================================
\section*{Unidad 2 Álgebra y Cálculo Tensorial}

\begin{exercise}{2.1} 
Las ecuaciones de transformación entre las coordenadas polares $(r, \theta)$ y las coordenadas cartesianas $(x, y)$ en un plano son:
%\begin{gather}
\begin{align*}
x &= r\cos\theta\\
y &= r\sin\theta
\end{align*}
%\quad \text{}
%\tag{1}
%\\[6pt]
\begin{align*}
r &= \sqrt{x^{2}+y^{2}}\\
\theta &= \tan^{-1}\!\left(\frac{y}{x}\right)
\end{align*}
%\quad \text{}
%\tag{2}
%\end{gather}
Calcula las cuatro transformadas parciales $\partial{x^{\mu}}/\partial{x'^{\nu}}$. 
\end{exercise}

\begin{exercise}{2.2} 
Datas las ecuaciones de transformación entre las coordenadas cartesianas y las coordendadas polares (ver Ejercicio 2.1), considera la función escalar $\Phi = bxy$ y calcula su gradiente en ambos sistemas de referencia.
\end{exercise}

\begin{exercise}{2.3}
Basándote en los resultados del Ejercicio 2.2, transforma las coordenadas cartesianas del vector $\nabla \Phi$ a coordenadas polares según las reglas de la transformación covariante de un vector. Comprueba que el resultado de dicha transformación coincide con las coordenadas polares del vector $\nabla \Phi$.
\end{exercise}

\begin{exercise}{2.4}
Considera el sistema de coordenadas parabólicas $(p,q)$ y sus transformaciones desde coordenadas cartesianas ordinarias $(x,y)$ dadas por:
\begin{align*}
p(x,y) &= x\\
q(x,y) &= y - c x^{2}
\end{align*}
donde $c$ es una constante. Prueba que las ecuaciones de transformación inversa son:
\begin{align*}
x(p,q) &= p\\
y(p,q) &= c p^{2} + q
\end{align*}

\begin{figure}[H]
\centering
\begin{tikzpicture}
\begin{axis}[
  axis lines=none,
  xmin=-4.5, xmax=4.5,
  ymin=-3,   ymax=5,      % rango vertical hasta q=5
  width=11cm, height=7cm,
  clip=true
]

% ----- parámetro c -----
\def\c{0.05}

% ====== Grid ======
% líneas p = const
\pgfplotsinvokeforeach{-4,-3,-2,-1,0,1,2,3,4}{
    \addplot[gray!70, thin, domain=-3:5] ({#1},{x});
}

% parábolas q = const
\pgfplotsinvokeforeach{-3,-2,-1,0,1,2,3,4,5}{
    \addplot[gray!70, thin, domain=-4.5:4.5] ({x},{\c*x^2 + #1});
}

% ====== Etiquetas SOLO en p=0 y q=0 ======
%\node[gray!70] at (axis cs:0,-2.8) {\scriptsize $p=0$};
%\node[gray!70, anchor=east] at (rel axis cs:0.995,{(0+4)/8}) {\scriptsize $q=0$};
% Nota: mapeo (q+3)/(qmax-qmin) con qmin=-3, qmax=5 → (0+3)/8=0.375

% ====== Vectores y puntos ======
% Punto O: (p,q)=(0,0)
\def\pO{0}\def\qO{0}
\pgfmathsetmacro{\xO}{\pO}
\pgfmathsetmacro{\yO}{\c*(\pO*\pO)+\qO}
\draw[-{Latex[length=3mm]}, thick] (axis cs:\xO,\yO) -- (axis cs:{\xO+1},{\yO+2*\c*\pO})
  node[pos=1, above] {$\mathbf{e}_p$};
\draw[-{Latex[length=3mm]}, thick] (axis cs:\xO,\yO) -- (axis cs:\xO,{\yO+1})
  node[pos=1, right] {$\mathbf{e}_q$};
\fill (axis cs:\xO,\yO) circle (1.2pt);
\node[below left] at (axis cs:\xO,\yO) {\small $\mathbf{O}$};

% Punto P: (p,q)=(2,2)
\def\pP{2}\def\qP{2}
\pgfmathsetmacro{\xP}{\pP}
\pgfmathsetmacro{\yP}{\c*(\pP*\pP)+\qP}
\draw[-{Latex[length=3mm]}, thick] (axis cs:\xP,\yP) -- (axis cs:{\xP+1},{\yP+2*\c*\pP})
  node[pos=1, above right] {$\mathbf{e}_p$};
\draw[-{Latex[length=3mm]}, thick] (axis cs:\xP,\yP) -- (axis cs:\xP,{\yP+1})
  node[pos=1, right] {$\mathbf{e}_q$};
\fill (axis cs:\xP,\yP) circle (1.2pt);
\node[below right] at (axis cs:\xP,\yP) {\small $\mathbf{P}$};

\end{axis}
\end{tikzpicture}
\caption{Coordenadas parabólicas con $p=x$ y $q=y-cx^2$}
\end{figure}

\end{exercise}

\begin{exercise}{2.5}
Dadas las ecuaciones de transformación entre coordenadas cartesianas y parabólicas del Ejercicio 2.4, y sabiendo que las componentes del tensor de la métrica en coordenadas cartesianas son las de la matriz identidad, i.e. $ds^2=dx^2+dy^2$, calcula el tensor de la métrica en coordenadas parabólicas $(p,q)$.    
\end{exercise}

\begin{exercise}{2.6}
Sea un vector \textbf{A} con componentes en coordenadas parabólicas (p, q): $A^p=1,\,A^q=0$. De acuerdo con las transformaciones entre las coordenadas parabólicas y las cartesianas especificadas en el Ejercicio 2.4:
\\(a) Encuentra las componentes de este vector en el sistema de coordenadas cartesianas (x,y).
\\(b) ¿Tienen sentido estas componentes? (Pista: dibuja los vectores base $\textbf{e}_p$ y $\textbf{e}_q$ en un punto típico).
\\(c) Demuestra que $A^2=\textbf{A}\cdot\textbf{A}$ tiene el mismo valor en ambos sistemas.  
\end{exercise}

\begin{exercise}{2.7}
Dadas las transformaciones entre coordenadas cartesianas $(x, y)$ y coordenadas polares $(r, \theta)$ del Ejercicio 2.1, considera el vector \textbf{V} con componentes en el sistema de referencia cartesiano $V^x=1$ y $V^y=0$ y calcula las componentes covariantes tanto en coordenadas cartesianas como polares. Comprueba que $V'^\mu V'_\nu = 1$ e interpreta el resultado.
\end{exercise}

\begin{exercise}{2.8}
Por definición, la delta de Kronecker $\delta^{\mu}{}_{\nu}$ tiene los mismos componentes en todos los sistemas de referencia ($\delta^{\mu}{}_{\nu}=1$ si $\mu=\nu$, y cero en caso contrario). Prueba que la delta de Kronecker es un tensor de tipo (2,1), es decir, sus coordenadas se transforman de acuerdo con la siguente expresión:
\[
\frac{\partial{x'^\mu}}{\partial{x^\alpha}}\frac{\partial{x^\alpha}}{\partial{x'^\nu}}\delta^{\alpha}{}_{\beta}=\delta'^{\mu}{}_{\nu}\equiv\delta^{\alpha}{}_{\beta}
\]
\end{exercise}

\begin{exercise}{2.9}
Prueba que la suma de dos tensores $A$ y $B$ de tipo (2, 1), es otro tensor $C$ de tipo (2,1), tal que sus componentes $C^{\mu\nu}{}{}_{\alpha}=A^{\mu\nu}{}{}_{\alpha} + B^{\mu\nu}{}{}_{\alpha}$ satisfacen la regla de transformación de los tensores.
\end{exercise}

\begin{exercise}{2.10}
Prueba que el product tensorial de dos tensores $A$ y $B$ de tipo (0, 2) y (1, 0) respectivamente, es otro tensor $C$ de tipo (1,2), tal que sus componentes $C_{\mu\nu}{}^{\alpha}=A_{\mu\nu}B^{\alpha}$ satisfacen la regla de transformación de los tensores.
\end{exercise}

\begin{exercise}{2.11}
\end{exercise}

\begin{exercise}{2.12}
\end{exercise}

\begin{exercise}{2.13}
\end{exercise}

\begin{exercise}{2.14}
\end{exercise}

\begin{exercise}{2.15}
Calcula las derivadas parciales $\partial{x^{\mu}}/\partial{x'^{\nu}}$ de las transformaciones de Lorentz, donde $(x^0, x^1, x^2, x^3)=(t,x,y,z)$:
\begin{align*}
t' & =\gamma(t-vx)\\
x' & =\gamma(x-vt)\\
y' & =y\\
z' & =z
\end{align*}
\begin{align*}
t & =\gamma(t'+vx')\\
x & =\gamma(x'+vt')\\
y & =y'\\
z & =z'
\end{align*}
y comprueba que se cumple la identidad:
\[
\frac{\partial{x'^\mu}}{\partial{x^\alpha}}\frac{\partial{x^\alpha}}{\partial{x'^\nu}}=\delta^{\mu}{}_{\nu}
\]
\end{exercise}

\begin{exercise}{2.16}
Comprueba que la métrica de Minkowski $\eta_{\mu\nu}=diag(-1, 1, 1, 1)$ es un tensor covariante de rango 2, i.e.
\[
\eta'_{\mu_\nu}=\frac{\partial{x^\alpha}}{\partial{x'^\mu}}\frac{\partial{x^\beta}}{\partial{x'^\nu}}\eta_{\alpha\beta}
\]
en los siguientes casos:
\\(a)\,$\mu=\nu=x^0$
\\(b)\,$\mu=x^0$\,;\,$\nu=x^1$
\\(c)\,$\mu=\nu=x^1$
\end{exercise}
\end{document}