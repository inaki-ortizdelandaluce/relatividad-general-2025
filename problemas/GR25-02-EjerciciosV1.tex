% --------------------------------------------------------------
% This is all preamble stuff that you don't have to worry about.
% Head down to where it says "Start here"
% --------------------------------------------------------------
 
\documentclass[12pt]{article}
 
\usepackage[margin=1in]{geometry} 
\usepackage{amsmath,amsthm,amssymb}
\usepackage[spanish]{babel} 
\usepackage[T1]{fontenc}

\newcommand{\N}{\mathbb{N}}
\newcommand{\Z}{\mathbb{Z}}
 
\newenvironment{theorem}[2][Teorema]{\begin{trivlist} 
\item[\hskip \labelsep {\bfseries #1}\hskip \labelsep {\bfseries #2.}]}{\end{trivlist}}
\newenvironment{lemma}[2][Lema]{\begin{trivlist} 
\item[\hskip \labelsep {\bfseries #1}\hskip \labelsep {\bfseries #2.}]}{\end{trivlist}}
\newenvironment{exercise}[2][Ejercicio]{\begin{trivlist} 
\item[\hskip \labelsep {\bfseries #1}\hskip \labelsep {\bfseries #2.}]}{\end{trivlist}}
\newenvironment{problem}[2][Problema]{\begin{trivlist} 
\item[\hskip \labelsep {\bfseries #1}\hskip \labelsep {\bfseries #2.}]}{\end{trivlist}}
\newenvironment{question}[2][Pregunta]{\begin{trivlist} 
\item[\hskip \labelsep {\bfseries #1}\hskip \labelsep {\bfseries #2.}]}{\end{trivlist}}
\newenvironment{corollary}[2][Corolario]{\begin{trivlist} 
\item[\hskip \labelsep {\bfseries #1}\hskip \labelsep {\bfseries #2.}]}{\end{trivlist}}

\newenvironment{solution}{\begin{proof}[Solución]}{\end{proof}} 
 
\begin{document}
 
% --------------------------------------------------------------
%                         Start here
% --------------------------------------------------------------
 
\title{Ejercicios de Relatividad General y Cosmología}
\author{Iñaki Ortiz de Landaluce\\ 
Introducción a la Relatividad General y Cosmología, Curso 2025-2026}
\date{} 
\maketitle

% ==============================================================
%                     UNIDAD 1: INTRODUCCIÓN
% ==============================================================
\section*{Unidad 2 Álgebra y Cálculo Tensorial}

\begin{exercise}{2.1} 
Las ecuaciones de transformación entre las coordenadas polares $r$, $\theta$ y las coordenadas cartesianas $x$, $y$ en un plano son:
\begin{gather}
\begin{aligned}
x &= r\cos\theta\\
y &= r\sin\theta
\end{aligned}
\quad \text{}
\tag{1}
\\[6pt]
\begin{aligned}
r &= \sqrt{x^{2}+y^{2}}\\
\theta &= \tan^{-1}\!\left(\frac{y}{x}\right)
\end{aligned}
\quad \text{}
\tag{2}
\end{gather}
Calcula las cuatro transformadas parciales $\partial{x^{\mu}}/\partial{x'^{\nu}}$. 
\end{exercise}

\begin{exercise}{2.2} 
Datas las ecuaciones de transformación entre las coordenadas cartesianas y las coordendadas polares (ver Ejercicio 2.1), considera la función escalar $\Phi = bxy$ y calcula su gradiente en ambos sistemas de referencia.
\end{exercise}

\begin{exercise}{2.3}
Basándote en los resultados del Ejercicio 2.2, transforma las coordenadas cartesianas del vector $\nabla \Phi$ a coordenadas polares según las reglas de la transformación covariante de un vector. Comprueba que el resultado de dicha transformación coincide con las coordenadas polares del vector $\nabla \Phi$.
\end{exercise}

\end{document}